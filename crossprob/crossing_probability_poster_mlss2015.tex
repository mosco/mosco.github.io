%%%%%%%%%%%%%%%%%%%%%%%%%%%%%%%%%%%%%%%%%
% a0poster Portrait Poster
% LaTeX Template
% Version 1.0 (22/06/13)
%
% The a0poster class was created by:
% Gerlinde Kettl and Matthias Weiser (tex@kettl.de)
% 
% This template has been downloaded from:
% http://www.LaTeXTemplates.com
%
% License:
% CC BY-NC-SA 3.0 (http://creativecommons.org/licenses/by-nc-sa/3.0/)
%
%%%%%%%%%%%%%%%%%%%%%%%%%%%%%%%%%%%%%%%%%

%----------------------------------------------------------------------------------------
%       PACKAGES AND OTHER DOCUMENT CONFIGURATIONS
%----------------------------------------------------------------------------------------

\documentclass[a0,portrait]{a0poster}

\usepackage{multicol} % This is so we can have multiple columns of text side-by-side

\usepackage[svgnames]{xcolor} % Specify colors by their 'svgnames', for a full list of all colors available see here: http://www.latextemplates.com/svgnames-colors

%\usepackage{times} % Use the times font
%\usepackage{palatino} % Palatino font
\usepackage{newcent} % New century schoolbook

\usepackage{graphicx} % Required for including images
\graphicspath{{figures/}} % Location of the graphics files
\usepackage{booktabs} % Top and bottom rules for table
\usepackage[font=small,labelfont=bf]{caption} % Required for specifying captions to tables and figures
\usepackage{amsfonts, amsmath, amsthm, amssymb} % For math fonts, symbols and environments
\usepackage{wrapfig} % Allows wrapping text around tables and figures
\usepackage{pstricks}
\usepackage{pst-plot}
\usepackage{xcolor}

\usepackage{framed, color}
\colorlet{shadecolor}{Orange!60}
\usepackage{float}

\usepackage{geometry}
\geometry{
    a0paper,
    top=4cm,
    bottom=2cm,
    left=7cm,
    right=2cm
}
\setlength{\marginparwidth}{0pt}

\DeclareMathOperator*{\argmin}{\arg\!\min}
\DeclareMathOperator*{\argmax}{\arg\!\max}
\DeclareMathOperator*{\expect}{\mathbb{E}}
\DeclareMathOperator*{\var}{\mathrm{Var}}

%\definecolor{bgcolor}{HTML}{FFF8DC} % Corn silk
%\definecolor{bgcolor}{HTML}{FFFAFA} % Snow
%\definecolor{bgcolor}{HTML}{FFFAF0} % Floral white
%\definecolor{bgcolor}{HTML}{F3ECDA} % Some type of slate
%\definecolor{color1dark}{HTML}{408c65} % Teal
%\definecolor{color1bright}{HTML}{b2e6cb} % Teal

% Blues
%\definecolor{bgcolor}{HTML}{dee7ee}
%\definecolor{color1dark}{HTML}{2A3132}
%\definecolor{color1bright}{HTML}{A6BFD1}

% Offwhite background with teal and orange
%\definecolor{bgcolor}{HTML}{FBF9F4}
\definecolor{bgcolor}{HTML}{F6F2E8}
%\definecolor{bgcolor}{HTML}{F3EDDA}
\definecolor{textcolor}{HTML}{65635d}
\definecolor{brighttextcolor}{HTML}{C9C8C5}
%\definecolor{emphtextcolor}{HTML}{1e1e1c}
\definecolor{emphtextcolor}{HTML}{000000}
\definecolor{color1dark}{HTML}{20948B}
\definecolor{color1bright}{HTML}{B2E6CB}
\definecolor{color2bright}{HTML}{FFB9A3}

\columnsep=150pt % This is the amount of white space between the columns in the poster
\columnseprule=0.5pt % This is the thickness of the black line between the columns in the poster
\renewcommand{\columnseprulecolor}{\color{textcolor}}

\DeclareCaptionFont{captiontitlecolor}{\color{emphtextcolor}}
\DeclareCaptionFont{captiontextcolor}{\color{textcolor}}
\captionsetup{labelfont={captiontitlecolor}, textfont={captiontextcolor}}

\newcommand{\COMMENTOUT}[1]{}
\newcommand{\x}{{\bf x}}
\renewcommand{\emph}[1]{{\color{emphtextcolor}#1}}
\newcommand{\simiid}{\overset{i.i.d.}{\sim}}
\newcommand{\ntoinf}{n \rightarrow \infty}
\newcommand{\pr}[1]{\Pr \left[ #1 \right]}


\newcommand{\framedblock}[1]{\begin{framed}#1\end{framed}}
\newcommand{\shadedblock}[1]{\begin{shaded}#1\end{shaded}}
%\newcommand{\orangeblock}[1]{\colorlet{shadecolor}{Orange!60} \shadedblock{#1}}
\newcommand{\blueblock}[1]{
    \colorlet{shadecolor}{Blue!10}
    \shadedblock{#1}
}
\newcommand{\grayblock}[1]{
    \colorlet{shadecolor}{Grey!25}
    \shadedblock{#1}
}
\newcommand{\yellowblock}[1]{
    \colorlet{shadecolor}{Yellow!35}
    \shadedblock{#1}
}
\newcommand{\centered}[1]{\begin{center}#1\end{center}}

\newcommand{\theoremblock}[1]{\blueblock{\noindent \emph{Theorem.} #1}}
\newcommand{\lemmablock}[1]{\colorlet{shadecolor}{color1bright}\shadedblock{\noindent \emph{Lemma.}#1}}
\newcommand{\emphblock}[1]{\colorlet{shadecolor}{color2bright}\shadedblock{\noindent #1}}
\newcommand{\sectiontitle}[1]{\section*{\huge \color{textcolor}#1}}

\newtheorem{definition}{Definition}[section] 
\newtheorem{theorem}{Theorem}[section] 
\newtheorem{lemma}[theorem]{Lemma}
\newtheorem{corollary}[theorem]{Corollary}
\newtheorem{remark}{Remark}[section]
\newtheorem{example}{Example}

\begin{document}

\pagecolor{bgcolor}
%----------------------------------------------------------------------------------------
%       POSTER HEADER 
%----------------------------------------------------------------------------------------

% The header is divided into two boxes:
% The first is 75% wide and houses the title, subtitle, names, university/organization and contact information
% The second is 25% wide and houses a logo for your university/organization or a photo of you
% The widths of these boxes can be easily edited to accommodate your content as you see fit
\begin{minipage}[b]{0.799\linewidth}\ \end{minipage}
\begin{minipage}[b]{0.5\linewidth}
\includegraphics[width=15cm]{wiz_logo1-01.pdf}\\
\end{minipage}
%\color{color1dark}
\color{textcolor}
\noindent {\fontsize{121}{140}\selectfont Fast computation of boundary crossing probabilities for the empirical CDF\\\\
%\noindent \VeryHuge{Boundary crossing probabilities for the empirical CDF}\\[0.3cm]
%%\noindent {\VERYHuge Fast algorithms for computing p-values of sup-type goodness of fit statistics}\\[1.8cm]
%\noindent {\fontsize{106}{120}\selectfont Fast algorithms for computing the boundary crossing probability of the empirical CDF}
%\noindent {\fontsize{85}{96}\selectfont The boundary crossing probability of the empirical CDF}\\[0.3cm]
%\noindent {\fontsize{85}{96}\selectfont Boundary crossing probabilities for the empirical CDF}\\[0.3cm]
%\noindent \Huge{fast algorithms for exact computation and their statistical applications}\\[0.8cm]
\color{textcolor} \large \textbf{Amit Moscovich-Eiger, Boaz Nadler}, Weizmann Institute of Science, Israel.\\
amit.moscovich@weizmann.ac.il; boaz.nadler@weizmann.ac.il\\

% \begin{minipage}[b]{0.55\linewidth}
%     \VeryHuge \color{NavyBlue} \textbf{Tail-sensitive goodness-of-fit}\\
%     \color{Black} % Title
%     \Huge{The Calibrated Kolmogorov-Smirnov test} % Subtitle
%     %\\[3cm] 
%     %\Large \textbf{Amit Moscovich Eiger, Boaz Nadler}\quad \Large Weizmann Institute of Science\\[0.4cm] % University/organization
%     %\Large \textbf{Clifford Spiegelman} \quad \Large Texas A\&M University\\[0.4cm] % University/organization
% \end{minipage}
% \begin{minipage}{0.5\linewidth}
%     \large \textbf{Amit Moscovich Eiger, Boaz Nadler}\ \large Weizmann Institute of Science\\[0.4cm]
% % University/organization
%     \large \textbf{Clifford Spiegelman} \ \large Texas A\&M University\\[0.8cm] % University/organization
% \end{minipage}

%\vspace{2cm} % A bit of extra whitespace between the header and poster content
\normalsize
%----------------------------------------------------------------------------------------

\begin{multicols}{2} % This is how many columns your poster will be broken into, a portrait poster is generally split into 2 columns
\large % Main text size.

%----------------------------------------------------------------------------------------
\sectiontitle{Problem setup}

Let $\hat{F}_n$ be the empirical CDF of $n$ draws from $U[0,1]$ w.l.o.g.
Given two functions $g, h:\mathbb{R} \to \mathbb{R}$,
compute the non-crossing probability
\begin{align} \label{eq:pr_cdf_no_cross}
    \pr{\forall t: g(t) < \hat{F}_n(t) < h(t)}
\end{align}
\emph{Several algorithms have been proposed over the years, all are $O(n^3)$.} [Epanechnikov 1968, Steck 1971, No\'e 1972, Friedrich \& Schellhaas 1998, Khmaladze \& Shinjikashvili 2001].
\subsection*{Equivalent formulation}
Let $U_{1:n} \le U_{2:n} \le \ldots \le U_{n:n}$ be the order statistics of $n$ draws from $U[0,1]$.
Given arbitrary bounds $b_1, \ldots, b_n, B_1, \ldots, B_n \in \mathbb{R}$ compute the probability
\begin{align} \label{eq:equivalent_formulation}
    \pr{\forall i: b_i < U_{i:n} < B_i}.
\end{align}
%----------------------------------------------------------------------------------------
\sectiontitle{Two-sided $O(n^3)$ algorithm [F\&S 1998]}
\lemmablock{
    $\hat{F}_n$
    satisfies $g(t) < \hat{F}_n(t) < h(t)$ for all $t$ if and only if it satisfies
    these inequalities at all times when $n \cdot g(t)$ or $n \cdot h(t)$ cross an integer.
}
\emph{Definition.} For any $s \in [0,1]$ and any $ m \in \{0, 1, 2, \ldots\}$, let
%$R(s,m)$ be the probability that $\hat{F}_n(s) = m/n$
%and that $\hat{F}_n(t)$  does not cross the boundaries $g(t), h(t)$ up to time $s$. i.e.
\[
    R(s,m) := \pr{\forall t \in [0,s]: g(t) < \hat{F}_n(t) < h(t) \text{ and } \hat{F}_n(s) = \tfrac{m}{n}}.
\]\\[-1.2cm]
\emph{Recursion relations.}
Let $0 = t_0 \le t_1 \le \ldots \le t_N = 1$ denote the sorted set of integer-crossing times of
$n \cdot g(t)$ and $n \cdot h(t)$.
%and let $Z_{\ell,i} \sim \text{Binomial}(n-\ell, \tfrac{t_{i+1} - t_i}{1-t_i})$.
The Chapman-Kolmogorov equations give the  recursion  relations of \cite{FriedrichSchellhaas1998}:
\begin{align*} \label{eq:Rsm}
    R(t_{i+1},m)
    =
    \begin{cases}
        \sum_{\ell} R(t_i, \ell) \cdot \pr{(t_i, \ell) \to (t_{i+1}, m)} & \text{if } g(t_{i+1}) < m/n < h(t_{i+1}) \\
        0 & \text{otherwise.}
    \end{cases}
\end{align*}
where $\pr{(t_i, \ell) \to (t_{i+1}, m)} = \pr{\text{Binomial}(n-\ell, \tfrac{t_{i+1} - t_i}{1-t_i}) = m-\ell}$.
\\
\emph{Solution.} Eq. \eqref{eq:pr_cdf_no_cross} is equal to $R(1,n)$, which can be computed in $O(n^3)$.\\

\begin{center}
    \begin{pspicture*}(-1,-1)(16,8)
        \psframe(0,0)(16,8)

        \psgrid[gridlabels=0, subgriddiv=0, gridcolor=black!20](0,0)(16,8)
        
        \psset{fillcolor=color2bright, fillstyle=solid, fillcolor=brighttextcolor}
        \pscurve(0,1.5)(10,5)(14,10)(0,8)
        \pscurve(5,0)(25,4)(16,0)
        
        \psset{linecolor=textcolor, linewidth=4pt, fillstyle=none}
        \psline(0,0)(0.78,0)(0.78,1)(3.4,1)(3.4,2)(7.5,2)(7.5,3)(8.57,3)(8.57,4)(9.42,4)(9.42,5)(11,5)(11,6)(13.96,6)(13.96,7)(15.58,7)(15.58,8)(16,8)

        % Y axis labels
        \rput(-0.5,0.13){\tiny $0$}
        \rput(-0.5,1){\tiny $1/8$}
        \rput(-0.5,2){\tiny $2/8$}
        \rput(-0.5,3){\tiny $3/8$}
        \rput(-0.5,4){\tiny $4/8$}
        \rput(-0.5,5){\tiny $5/8$}
        \rput(-0.5,6){\tiny $6/8$}
        \rput(-0.5,7){\tiny $7/8$}
        \rput(-0.5,7.85){\tiny $1$}

        % Y axis ticks
        \psset{linecolor=black, linewidth=1pt}
        \qline(-0.1,0.02)(0.1,0.02)
        \qline(-0.1,1)(0.1,1)
        \qline(-0.1,2)(0.1,2)
        \qline(-0.1,3)(0.1,3)
        \qline(-0.1,4)(0.1,4)
        \qline(-0.1,5)(0.1,5)
        \qline(-0.1,6)(0.1,6)
        \qline(-0.1,7)(0.1,7)
        \qline(-0.1,8)(0.1,8)
   
        % X axis labels     
        \rput(0,-0.5){\tiny $t_0$}
        \rput(2.2,-0.5){\tiny $t_1$}
        \rput(5,-0.5){\tiny $t_2$}
        \rput(5.6,-0.5){\tiny $t_3$}
        \rput(7.05,-0.5){\tiny $t_4$}
        \rput(8.2,-0.5){\tiny $t_5$}
        \rput(9.25,-0.5){\tiny $t_6$}
        \rput(10,-0.5){\tiny $t_7$}
        \rput(11.32,-0.5){\tiny $t_8$}
        \rput(11.75,-0.5){\tiny $t_9$}
        \rput(12.7,-0.5){\tiny $t_{10}$}
        \rput(13.65,-0.5){\tiny $t_{11}$}
        \rput(14.3,-0.5){\tiny $t_{12}$}
        \rput(15.83,-0.5){\tiny $t_N$}
        
        % X axis ticks
        \psset{linecolor=black, linewidth=1pt}
        \qline(0,-0.1)(0,0.1)
        \qline(2.2,-0.1)(2.2,0.1)
        \qline(5,-0.1)(5,0.1)
        \qline(5.6,-0.1)(5.6,0.1)
        \qline(7.05,-0.1)(7.05,0.1)
        \qline(8.2,-0.1)(8.2,0.1)
        \qline(9.25,-0.1)(9.25,0.1)
        \qline(10,-0.1)(10,0.1)
        \qline(11.42,-0.1)(11.42,0.1)
        \qline(11.65,-0.1)(11.65,0.1)
        \qline(12.7,-0.1)(12.7,0.1)
        \qline(13.65,-0.1)(13.65,0.1)
        \qline(14.3,-0.1)(14.3,0.1)
        \qline(16,-0.1)(16,0.1)
  
        \psset{fillcolor=black, fillstyle=none, linewidth=0.09, linecolor=black}    
        %\pscircle(2.2,2){0.3}
        \qline(2.0,1.8)(2.4,2.2)
        \qline(2.0,2.2)(2.4,1.8)
        
        %\pscircle(5.6,3){0.3}
        \qline(5.42,2.8)(5.82,3.2)
        \qline(5.42,3.2)(5.82,2.8)
        
        %\pscircle(8.2,4){0.3}
        \qline(8.01,3.8)(8.41,4.2)
        \qline(8.01,4.2)(8.41,3.8)
        
        %\pscircle(10,5){0.2}
        \qline(9.8,4.8)(10.2,5.2)
        \qline(9.8,5.2)(10.2,4.8)
        
        %\pscircle(11.42,6){0.2}
        \qline(11.22,5.8)(11.62,6.2)
        \qline(11.22,6.2)(11.62,5.8)
        
        %\pscircle(12.7,7){0.1}
        \qline(12.5,6.8)(12.9,7.2)
        \qline(12.5,7.2)(12.9,6.8)
        
        %\pscircle(13.65,8){0.3}
        \qline(13.45,7.8)(13.85,8.2)
        \qline(13.45,8.2)(13.85,7.8)
        
        %\pscircle(5,0){0.3}
        \qline(4.8,-0.2)(5.2,0.2)
        \qline(4.8,0.2)(5.2,-0.2)
        
        %\pscircle(7.05,1){0.3}
        \qline(6.85,0.8)(7.25,1.2)
        \qline(6.85,1.2)(7.25,0.8)
                
        %\pscircle(9.25,2){0.3}
        \qline(9.05,1.8)(9.45,2.2)
        \qline(9.05,2.2)(9.45,1.8)
        
        %\pscircle(11.65,3){0.3}
        \qline(11.45, 2.8)(11.85, 3.2)
        \qline(11.45, 3.2)(11.85, 2.8)

        %\pscircle(14.3,4){0.3}
        \qline(14.1,3.8)(14.5,4.2)
        \qline(14.1,4.2)(14.5,3.8)
    \end{pspicture*}
    \ 
    \begin{pspicture*}(-1,-1)(16,8)
        \psframe(0,0)(16,8)

        \psgrid[gridlabels=0, subgriddiv=0, gridcolor=black!20](0,0)(16,8)

        \psset{fillcolor=color2bright, fillstyle=solid, fillcolor=brighttextcolor}
        \pscurve(0,1.5)(10,5)(14,10)(0,8)
        \pscurve(5,0)(25,4)(16,0)

        % X axis labels     
        \rput(0,-0.5){\tiny $t_0$}
        \rput(2.2,-0.5){\tiny $t_1$}
        \rput(5,-0.5){\tiny $t_2$}
        \rput(5.6,-0.5){\tiny $t_3$}
        \rput(7.05,-0.5){\tiny $t_4$}
        \rput(8.2,-0.5){\tiny $t_5$}
        \rput(9.25,-0.5){\tiny $t_6$}
        \rput(10,-0.5){\tiny $t_7$}
        \rput(11.32,-0.5){\tiny $t_8$}
        \rput(11.75,-0.5){\tiny $t_9$}
        \rput(12.7,-0.5){\tiny $t_{10}$}
        \rput(13.65,-0.5){\tiny $t_{11}$}
        \rput(14.3,-0.5){\tiny $t_{12}$}
        \rput(15.83,-0.5){\tiny $t_N$}
        
        % X axis ticks
        \psset{linecolor=black, linewidth=1pt}
        %\qline(0,-0.1)(0,0.1)
        %\qline(2.2,-0.1)(2.2,0.1)
        %\qline(5,-0.1)(5,0.1)
        \qline(5.6,-0.1)(5.6,0.1)
        \qline(7.05,-0.1)(7.05,0.1)
        \qline(8.2,-0.1)(8.2,0.1)
        \qline(9.25,-0.1)(9.25,0.1)
        \qline(10,-0.1)(10,0.1)
        \qline(11.42,-0.1)(11.42,0.1)
        \qline(11.65,-0.1)(11.65,0.1)
        \qline(12.7,-0.1)(12.7,0.1)
        \qline(13.65,-0.1)(13.65,0.1)
        \qline(14.3,-0.1)(14.3,0.1)
        \qline(16,-0.1)(16,0.1)

        \psset{fillcolor=textcolor, fillstyle=solid, linewidth=1pt, linecolor=black}    

        \psset{fillstyle=solid}
        \pscircle(0,0){0.15}

        \psline[arrowsize=0.3]{->}(0.2,0.03)(2,0.03)
        \psline[arrowsize=0.3]{->}(0.2,0.1)(2,0.9)
        
        \psset{fillstyle=solid}
        \pscircle(2.2,0.0){0.15}
        \pscircle(2.2,1){0.15}
        \psset{fillstyle=none}    
        \pscircle(2.2,2){0.15}

        \psline[arrowsize=0.3]{->}(2.4,0.08)(4.8,0.9)
        \psline[arrowsize=0.3]{->}(2.4,0.16)(4.8,1.8)
        \psline[arrowsize=0.3]{->}(2.4,1)(4.8,1)
        \psline[arrowsize=0.3]{->}(2.4,1.06)(4.8,1.95)


        \psset{fillstyle=none}    
        \pscircle(5,0){0.15}
        \psset{fillstyle=solid}
        \pscircle(5,1){0.15}
        \pscircle(5,2){0.15}

        \psset{fillstyle=solid}
        \pscircle(5.6,1){0.15}
        \pscircle(5.6,2){0.15}
        \psset{fillstyle=none}    
        \pscircle(5.6,3){0.15}

        \psset{fillstyle=none}    
        \pscircle(7.05,1){0.15}
        \psset{fillstyle=solid}
        \pscircle(7.05,2){0.15}
        \pscircle(7.05,3){0.15}

        \pscircle(8.2,2){0.15}
        \pscircle(8.2,3){0.15}
        \psset{fillstyle=none}    
        \pscircle(8.2,4){0.15}

        \psset{fillstyle=none}    
        \pscircle(9.25,2){0.15}
        \psset{fillstyle=solid}
        \pscircle(9.25,3){0.15}
        \pscircle(9.25,4){0.15}

        \psset{fillstyle=solid}
        \pscircle(10,3){0.15}
        \pscircle(10,4){0.15}
        \psset{fillstyle=none}    
        \pscircle(10,5){0.15}

        \psset{fillstyle=solid}
        \pscircle(11.42,3){0.15}
        \pscircle(11.42,4){0.15}
        \pscircle(11.42,5){0.15}
        \psset{fillstyle=none}    
        \pscircle(11.65,3){0.15}
        \psset{fillstyle=solid}
        \pscircle(11.65,4){0.15}
        \pscircle(11.65,5){0.15}
        \pscircle(11.65,6){0.15}
        \psset{fillstyle=none}    
        \pscircle(11.42,6){0.15}

        \psset{fillstyle=solid}
        \pscircle(12.7,4){0.15}
        \pscircle(12.7,5){0.15}
        \pscircle(12.7,6){0.15}
        \psset{fillstyle=none}    
        \pscircle(12.7,7){0.15}

        \psset{fillstyle=none}    
        \pscircle(13.65,8){0.15}
        \psset{fillstyle=solid}
        \pscircle(13.65,7){0.15}
        \pscircle(13.65,6){0.15}
        \pscircle(13.65,5){0.15}
        \pscircle(13.65,4){0.15}

        \psset{fillstyle=none}    
        \pscircle(14.3,4){0.15}
        \psset{fillstyle=solid}
        \pscircle(14.3,5){0.15}
        \pscircle(14.3,6){0.15}
        \pscircle(14.3,7){0.15}
        \pscircle(14.3,8){0.15}

        \psset{fillstyle=solid}
        \pscircle(16,5){0.15}
        \pscircle(16,6){0.15}
        \pscircle(16,7){0.15}
        \pscircle(16,8){0.15}

    \end{pspicture*}
    \captionof{figure}{(left panel) \emph{x} marks the $i/n$ crossing points.
    $\hat{F}_n$ crosses one of the boundaries if and only if it crosses an \emph{x} mark;
    (right panel) Layer graph representing the entries $R(t_i, m)$.}
\end{center}
%----------------------------------------------------------------------------------------
\sectiontitle{Two-sided $O(n^3)$ algorithm [K\&S 2001]}
\lemmablock{ The distribution of the stochastic process $n \cdot \hat{F}_n(t)$ is identical to that of a Poisson process $\xi_n(t)$ with intensity $n$  conditioned on $\xi_n(1) = n$.
}
\noindent \emph{Definition.} For any $s \in [0,1]$ and any $ m \in \{0, 1, 2, \ldots\}$, let
%$Q(s,m)$ be the probability that $\xi_n(s) = m$
%and that the scaled Poisson process $\tfrac1n \xi_n$ does not cross the boundaries $g(t), h(t)$ up to time $s$. i.e.
\[
    Q(s,m) := \pr{\forall t \in [0,s]: g(t) < \tfrac1n \xi_n(t) < h(t) \text{ and } \xi_n(s) = m}.
\]\\[-1.2cm]
\emph{Recursion relations.}
Similarly to the previous algorithm, the Chapman-Kolmogorov equations give
the recursive relations of \cite{KhmaladzeShinjikashvili2001}:
\begin{align*}
    Q(t_{i+1},m)
    =
    \begin{cases}
        \sum_{\ell} Q(t_i, \ell) \cdot\pr{Z_i = m-\ell} & \text{ if } g(t_{i+1}) < m/n < h(t_{i+1}) \\
        0 & \text{otherwise}
    \end{cases}
\end{align*}
where $Z_i$ is a Poisson random variable with intensity $n(t_{i+1} - t_i)$.\\
\emph{Solution.} Apply the lemma
to obtain 
\begin{align*}
    \pr{\forall t: g(t) < \hat{F}_n(t) < h(t)}
    &=
    Q(1,n) / \pr{\text{Poisson}(n)=n}.
\end{align*}
Computing Q(1,n) still requires $O(n^3)$ operations.
%----------------------------------------------------------------------------------------
\sectiontitle{New two-sided $O(n^2 \log n)$ algorithm}
Denote $Q_{t_i} = (Q(t_i,0), \ldots, Q(t_i,n))$ and $ \pi_\lambda = (\pr{Z_\lambda = 0}, \ldots, \pr{Z_\lambda = n})$
where $Z_\lambda$ is a Poisson random variable with expected value $\lambda$.
\emphblock{\emph{Key idea:} the vector $Q_{t_{i+1}}$ is nothing but a truncated linear convolution of  $Q_{t_i}$ and $\pi_{n(t_{i+1}-t_i)}$.
    Hence, using the circular convolution theorem and the Fast Fourier Transform
    we can compute $Q_{t_{i+1}}$ in $O(n \log n)$ time.
}

\begin{enumerate}
    \item Append $n$ zeros to the end of $Q_{t_i}$ and $\pi_{n(t_{i+1}-t_i)}$,
    forming $Q^{2n}$ and $\pi^{2n}$.
    \item Compute the FFT $\mathcal{F}\{Q^{2n}\}$ and $\mathcal{F}\{\pi^{2n}\}$.
    \item Apply the convolution theorem
    \(
        C^{2n} = \mathcal{F}\{Q^{2n} \star \pi^{2n}\} = \mathcal{F} \{ Q^{2n}\} \cdot \mathcal{F}\{\pi^{2n}\},
    \)
    where $\star$ denotes cyclic convolution and $\cdot$ is pointwise multiplication.
    \item Compute the inverse Fourier transform of $C^{2n}$ to obtain the vector $Q_{t_{i+1}}$
    \begin{align*}
        Q_{t_{i+1}}(m) =
        \begin{cases}
            \mathcal{F}^{-1}\{C^{2n}\}(m) & \text{if }  g(t_{i+1}) < m/n < h(t_{i+1}) \\
            0 & \text{otherwise}.
        \end{cases}
    \end{align*}
\end{enumerate}
Repeating this procedure for all $i$ yields a total running time of $O(n^2 \log n)$.
\emph{This is the fastest known algorithm for computing the two-sided crossing probability and the first to break the $O(n^3)$ barrier.}

%------------------------------------------------
\sectiontitle{New one-sided $O(n^2)$ algorithm}
In the one-sided case ($g<0$ or $h>1$) an even faster algorithm is possible. 
The joint density of the random vector of uniform order statistics is
\[
    f(U_{1:n}, \ldots, U_{n:n}) = 
    \left\{\begin{array}{ll}
        n! & \mbox{if } 0 \le U_{1:n} \le \ldots \le U_{n:n} \le 1, \\
        0  & \mbox{otherwise}. \\
    \end{array}
    \right.
\]
Hence the one-sided variant of Eq. \eqref{eq:equivalent_formulation} is given by
\begin{align*}
    \pr{\forall i: b_i < U_{i:n}}
    &=
    n!Vol\{ (U_{1:n}, \ldots, U_{n:n}) \ |\ \forall i: b_i < U_{i:n} \le U_{i+1:n}\} \nonumber \\
    &=  \displaystyle n! \int_{b_n}^1 dU_{n:n} \int_{b_{n-1:n}}^{U_{n:n}} dU_{(n-1)} \ldots \int_{b_2}^{U_{3:n}} dU_{2:n} \int_{b_1}^{U_{2:n}} dU_{1:n} \,.
\end{align*}
Numerically evaluating this integral from right to left takes \(O(n^{2})\) time.
A na\"ive implementation fails at $n \approx 150$ due to  numerical errors,
but with some effort we have been able to get up to $n \approx 50,000$. \cite{MoscovichEigerNadlerSpiegelman2015}

%------------------------------------------------
\sectiontitle{Application: p-value computation for goodness-of-fit statistics}
%\lemmablock{ This probability is equal to Eq. \eqref{eq:pr_cdf_no_cross}
%where $g(t)$ and $h(t)$ are taken to be the empirical CDFs of $\{B_1, \ldots, B_n\}$ and $\{b_1, \ldots, b_n\}$ respectively.
%}
\noindent The p-value of several sup-type continuous goodness-of-fit statistics directly translates to a probability
of the form of Eq. \eqref{eq:pr_cdf_no_cross}.
Hence we can compute such $p$-values in $O(n^2 \log n)$ time.
The following table demonstrates that this improvement is not merely theoretical but yields a significant reduction
in running times.\\
\begin{center}
    \centering
    \framedblock{
        \begin{tabular}{c c c c c} 
            \emph{Two-sided}    & $n=4000$ & $n=16,000$ & $n=64,000$ & $n=256,000$ \\ \hline
            \emph{K\&S 2001}          & 0.5 sec  & 8 sec      &  94 sec    & 18 minutes  \\
            \emph{$O(n^2 \log n)$ algorithm}         & 0.3 sec  & 2 sec      &  15 sec    & 117 sec \\[0.5cm]
            \emph{One-sided}&  \\\hline
            \emph{K\&S 2001}          & 45 sec   & 24 minutes & 18 hours   & weeks \\
            \emph{$O(n^2 \log n)$ algorithm}         & 2 sec    & 29 sec     & 9 minutes  & 3 hours\\
            \emph{$O(n^2)$ algorithm} & 1.3 sec  & 19 sec     & n/a        & n/a 
        \end{tabular}
    }
    \captionof{table}{Running times for computing $p$-values of the  $M_n$ goodness-of-fit statistics of Berk \& Jones.}
    \label{tbl:running_times}
\end{center}

%----------------------------------------------------------------------------------------
%\vspace{1cm}
%\section*{\color{black}\huge Applications}
%\begin{itemize}
%    \item Construction of confidence bands for empirical CDFs
%    \item Changepoint detection
%    \item Sequential testing
%    \item Brownian motion
%    \item Queuing theory 
%\end{itemize}
%----------------------------------------------------------------------------------------
\sectiontitle{Summary}
\emphblock{
\begin{itemize}
    \item State-of-the-art $O(n^2 \log n)$ algorithm for computing the two-sided crossing probability of empirical CDFs and Poisson processes.
    \item Fast $O(n^2)$ algorithm for the one-sided case.
    \item Potential applications include: p-value and power calculations for goodness-of-fit statistics, construction of $\alpha$-level confidence bands for distribution functions,  analysis of boundary crossing and first passage of a Brownian motion, queuing theory, sequential testing... 
    \item Efficient C++ code at: {\color{blue} http://www.wisdom.weizmann.ac.il/\textasciitilde amitmo}
\end{itemize}
}

%----------------------------------------------------------------------------------------
%\sectiontitle{Supporting material}
%The $O(n^2 \log n)$ algorithm is described in \cite{MoscovichEigerNadler2015}.
%For the one-sided $O(n^2)$ algorithm see Section 5 and Appendix C of \cite{MoscovichEigerNadlerSpiegelman2015}.
%Efficient C++ implementations are available at: {\color{blue} http://www.wisdom.weizmann.ac.il/\textasciitilde amitmo}
%----------------------------------------------------------------------------------------
\nocite{*} % Print all references regardless of whether they were cited in the poster or not
\normalsize
\bibliographystyle{unsrt} % Plain referencing style
%\bibliographystyle{elsarticle-num-names}
\bibliography{crossing-probability}



%----------------------------------------------------------------------------------------

\end{multicols}
\end{document}
